\documentclass[a4paper,oneside]{article}

\usepackage{xcolor}
\usepackage{amsmath,amssymb}
\usepackage{fancyhdr}
\usepackage{ctex}
\usepackage{xeCJK}
\usepackage{bookmark}
\usepackage{bookmark}
\usepackage{titling}
\usepackage{enumitem}

\setCJKmainfont{SimSun}[BoldFont=SimHei,ItalicFont=KaiTi]

\hypersetup{hidelinks}
\hypersetup{
pdftitle={在共同价值中思考中国担当}, % PDF标题
pdfauthor={崔俊熙}, % 作者
pdfcreator={LaTeX}, % 创建者
pdfproducer={XeLaTeX} % 制作工具
}

\title{在共同价值中思考中国担当}
\author{崔俊熙}
\date{\today}

\begin{document}

\pagestyle{fancy}
\fancyhf{}

\lhead{\today}
\chead{在共同价值中思考中国担当}
\rhead{崔俊熙}
\lfoot{}
\cfoot{\thepage}
\rfoot{}

\maketitle

\section{时代背景}

\subsection{全球断裂}

安全,发展,科技,伦理等多重层面出现结构性分裂,合作难度上升。

\subsection{价值失序}

不同文明间的价值体系缺乏沟通空间,``普遍主义僭越''冲突加剧。

\subsection{中国命题}

世界需要一种非排他,跨文明的共同价值框架,中国应当提出答案。

\section{共同价值}

\subsection{普遍性困境}

单一文明的普遍主义难以在多元世界取得合法性。

\subsection{共在性价值}

共同价值来自全球相互依存,是各文明的``重叠共识''。

\subsection{中国思想资源}

``和合''``天下''等传统强调差异中的整体性,为共同价值提供文化源泉。

\section{中国担当}

\subsection{文明逻辑}

中国现代化代表``多路径现代性'',拓展全球发展想象。

\subsection{伦理逻辑}

``义利合一''体现关系伦理,强调责任,互利与公共性。

\subsection{结构责任}

全球相互依存使中国利益自然外溢为全球责任。

\section{中国实践}

\subsection{理念贡献}

提出和平发展,科学发展的道路,探索非对抗性的现代化路径。

\subsection{制度贡献}

上合组织,WTO,APEC,二十国集团,77国集团和中国等构建开放多边合作平台。

\subsection{公共产品}

对外基础设施合作与产能合作,援建医疗和教育事业体现现实担当。

\section{全球意义}

\subsection{第三种选择}

应当是介于传统现实主义与西方价值霸权之间的合作范式。

\subsection{价值+制度}

既有价值深度,又有制度可行性。

\subsection{文明角色}

从国家担当走向文明担当,参与未来全球秩序重塑。

\end{document}