\documentclass[a4paper,oneside]{article}

\usepackage{xcolor}
\usepackage{amsmath,amssymb}
\usepackage{fancyhdr}
\usepackage{ctex}
\usepackage{xeCJK}
\usepackage{bookmark}
\usepackage{bookmark}
\usepackage{titling}
\usepackage{enumitem}

\setCJKmainfont{SimSun}[BoldFont=SimHei,ItalicFont=KaiTi]

\hypersetup{hidelinks}
\hypersetup{
pdftitle={在共同价值中思考中国担当}, % PDF标题
pdfauthor={崔俊熙}, % 作者
pdfcreator={LaTeX}, % 创建者
pdfproducer={XeLaTeX} % 制作工具
}

\title{在共同价值中思考中国担当}
\author{崔俊熙}
\date{\today}

\begin{document}

\pagestyle{fancy}
\fancyhf{}

\lhead{\today}
\chead{在共同价值中思考中国担当}
\rhead{崔俊熙}
\lfoot{}
\cfoot{\thepage}
\rfoot{}

\maketitle
\tableofcontents

\section*{引言}
各位老师、同学们:
今天我想与大家分享的题目是—``在共同价值中思考中国担当''。

当我们讨论中国的全球角色时,不仅在谈国家实力或经济增长,更是在回答一个更深的问题:
在多元文明并存、而全球又高度相互依存的时代,中国能向世界贡献什么样的价值与责任观念?

\section{时代背景}

当下的世界正经历多重断裂。

\begin{itemize}
    \item \textbf{经济断裂}:全球化遭遇逆流,各国以保护主义方式重塑产业链。
    \item \textbf{地缘断裂}:大国关系剧烈震荡,冷战集团化倾向再现。
    \item \textbf{文化断裂}:文明相互指责,排斥性叙事浮现,文化身份政治化。
\end{itemize}

这些断裂共同导致国际社会合作成本升高、信任下降,世界仿佛失去了共同语言。

不同文明之间缺乏稳定的沟通机制,价值体系相互排斥;一些国家试图用单一价值来框定世界,而其他国家则感到被排除在外。
价值之争从深层削弱了全球治理能力,也让世界陷入``合法性危机'':我们无法就什么是``共同的善''达成共识。
因此,我们需要寻找一种能够跨越文明差异、能够支撑全球合作的\textbf{共同价值}。

世界需要一种非排他,可对话,跨文明的共同价值框架,中国应当提出答案。

在这样的历史点位上,``中国担当''不再是一个国家自我叙事的问题,而是一个``世界如何继续合作''的时代命题。

\section{共同价值}

\subsection{普遍性困境}

传统的普遍主义通常以自身历史经验为全球模板,但这种方式在全球多样性日益凸显的时代显得力不从心。
\begin{itemize}
    \item 它忽视了历史文明的差异性;
    \item 缺乏全球代表性;
    \item 容易演变为文化优越感。
\end{itemize}

结果是:普遍性越来越被视为某种``文化扩张'',而非真正的跨人类共识。

\subsection{共在性价值}

中国提出的``共同价值''不是去构建一种新的普遍主义,而是强调:
\begin{itemize}
    \item 共性来自人类相互依存的现实,而不是来自单一文明的延伸;
    \item 差异不是障碍,而是构成共识的起点;
    \item 文明之间可以形成``重叠共识'',而不是统一化的价值模板。
\end{itemize}

因此我们今天讲的``共同价值'',不是要把世界塑造成单一模式,而是寻找不同文明之间可以共享的重叠共识:
\begin{itemize}
    \item 对和平的重视,
    \item 对发展的期待,
    \item 对公平与稳定的追求。
\end{itemize}

这些共识不是某个文明的专利,而是全球相互依存所共同塑造的价值。

\subsection{中国思想资源}

中国文明长期强调多元、整体、共生,为共同价值的提炼提供了深厚土壤:
\begin{itemize}
    \item ``和而不同'':承认差异、追求和谐,反对同质化;
    \item ``天下''观:以秩序与伦理组织世界,而非以征服或扩张为逻辑;
    \item ``中庸与互利'':重视均衡与合作,强调义利统一;
    \item 传统政治哲学中的``仁政''理念:将共同生活与公共福祉置于价值核心。
\end{itemize}

这些思想为全球共识提供了跨文化的表达方式,也为形成新的共同价值创造了思想基础。

\section{中国担当}

\subsection{文明逻辑:多路径现代性的形成}

改革开放以来,中国探索了一条不同于西方模式的现代化道路——国家主导、渐进改革、面向发展的全球参与。
这一道路证明:现代化可以是多样的,不必走单一轨道。
这本身就是对世界文明的一种贡献,拓展了其他国家的选择空间。

\subsection{伦理逻辑:关系型的责任观}

中国传统政治文化的核心伦理可以用``关系性''来概括:
\begin{itemize}
    \item 重视组织与个体之间的责任;
    \item 强调合作胜于竞争;
    \item 信奉利益与道义的结合,而非利益与道义的冲突。
\end{itemize}

传统儒家的``义利合一'',``推己及人''等理念,在现代国家政策中得到了不同程度的延续,使中国具备从伦理上理解全球责任的基础。

\subsection{结构逻辑:大国的外溢责任}

随着中国经济体量在改革开放后逐步扩大(尤其在 1980-2000 年全球化高峰期),中国与世界之间形成高度联系:
\begin{itemize}
    \item 世界的稳定影响中国的发展;
    \item 中国的发展也影响世界市场、供应链与区域平衡。
\end{itemize}

这种结构性嵌入意味着——
中国无法回避国际责任,也天然拥有推动全球公共利益的能力。

\section{中国实践}

\subsection{理念贡献:和平发展与合作共赢}

中国系统提出和平发展和科学发展的道路,强调国家之间可以在竞争中共存,在发展中共享利益。
这一理念成为中国对外关系的重要价值基调,也为其他国家探索非对抗性的现代化路径提供了参考。

\subsection{制度贡献:深化国际与地区合作机制}

中国积极推动和参与多边制度建设,包括:
\begin{itemize}
    \item 加入恢复联合国合法席位,推动多边主义重建;
    \item 加入世界贸易组织,促进全球贸易体系开放稳定;
    \item 推动上海合作组织,深化安全与经济合作;
    \item 加入世界银行、国际货币基金组织,推动发展议题纳入核心议程;
    \item 推动亚太地区合作,如APEC等;
    \item 推动金砖国家合作;
    \item 与发展中国家的组织77国集团深度合作,``77国集团和中国''称为国际上的焦点。
    \item 扩大与非洲、拉美的合作,如中非合作论坛。
\end{itemize}

这些制度贡献强调开放、合作与普遍参与,拒绝排他性结构。
这些开放多边合作机制为发展中国家参与全球秩序提供了制度平台,强化了南南合作的现实基础。

\subsection{公共产品贡献:以发展带动共同发展}

我国的公共产品实践更强调实际发展支持,如:
\begin{itemize}
    \item 大规模援助与减免最不发达国家债务;
    \item 推动对外基础设施合作与产能合作的早期探索;
    \item 帮助多国建设农业技术示范中心、医疗队、教育培训项目;
    \item 在国际金融危机(2008)中,中国保持投资和进口,为世界经济复苏提供需求动力。
\end{itemize}

这些行动并不以意识形态输出为目的,而是以发展作为公共产品直接惠及他国。

\subsection{行动模式}

中国行动呈现四个显著特征:
\begin{itemize}
    \item 以发展为中心:强调经济与社会改善是稳定的根本;
    \item 以合作为前提:不以意识形态划线,强调互利;
    \item 以经验为基础:提供可复制、可调整的方案;
    \item 以开放为方向:鼓励融入世界,而非形成排他体系。
\end{itemize}

这种模式为全球南方国家提供了另一种发展可能性。

\section{全球意义}

\subsection{提供``第三种选择''}

在``价值霸权''与``现实主义对抗''之外,中国提供的是一种结合和平、发展与合作的第三种治理思路。它强调:
\begin{itemize}
    \item 多元文明的共存;
    \item 不干涉原则与对话机制并存;
    \item 在风险社会中寻找共赢的可能性。
\end{itemize}
这为全球南方国家带来新的可能性。

\subsection{融合价值与制度的双重贡献}

中国经验的价值在于其 伦理基础与制度实践的结合:
\begin{itemize}
    \item 在伦理层面,强调合作、互利、公共性;
    \item 在制度层面,通过长期参与国际组织、区域合作机制、经济互补体系等方式不断贡献稳定性。
\end{itemize}

这种``双轨形式''提升了中国担当在全球治理中的可执行性和可持续性。

\subsection{从国家担当走向文明担当}

作为一个具有独特文明传统和现代化经验的大国,中国的责任不仅是经济和政治层面的,更是文明层面的:
\begin{itemize}
    \item 不是输出单一模式,而是共享经验;
    \item 不追求文化同质,而是倡导文明互补;
    \item 在重大历史转折点上,为全球秩序的重构提供多极、多元、多路径的思想资源。
\end{itemize}

中国可以在全球秩序变革中提供一种以多样性、公平性、合作性为基础的治理理念。

\end{document}