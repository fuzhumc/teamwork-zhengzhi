\documentclass[aspectratio=169]{ctexbeamer}
\usepackage[redtheme]{ustcbeamer}
\input{ustctheme.tex}

\title{在共同价值中思考中国担当}
\author{崔俊熙}
\institute[USTC]{中国科学技术大学}
\date{2025年11月24日}
\begin{document}

\maketitleframe

\begin{frame}
	\frametitle{大纲}
	\tableofcontents[hideallsubsections]
\end{frame}

\AtBeginSection[]{
\setbeamertemplate{footline}[footlineoff]
  \begin{frame}
    \frametitle{大纲}
    \tableofcontents[currentsection,subsectionstyle=show/show/hide]
  \end{frame}
\setbeamertemplate{footline}[footlineon]
}

\AtBeginSubsection[]{
\setbeamertemplate{footline}[footlineoff]
  \begin{frame}
    \frametitle{大纲}
    \tableofcontents[currentsection,subsectionstyle=show/shaded/hide]
  \end{frame}
\setbeamertemplate{footline}[footlineon]
}

\section{时代背景}

\begin{frame}
  \frametitle{时代背景}
  当下的世界正经历多重断裂。
  \begin{itemize}
      \item \textbf{经济断裂}:全球化遭遇逆流,各国以保护主义方式重塑产业链。
      \item \textbf{地缘断裂}:大国关系剧烈震荡,冷战集团化倾向再现。
      \item \textbf{文化断裂}:文明建相互指责,排斥性叙事浮现,文化身份政治化。
  \end{itemize}

  这些断裂共同导致国际社会合作成本升高、信任下降,世界仿佛失去了共同语言
\end{frame}

\begin{frame}
  \frametitle{时代背景}
  不同文明之间缺乏稳定的沟通机制,价值体系相互排斥;一些国家试图用单一价值来框定世界,而其他国家则感到被排除在外。
  价值之争从深层削弱了全球治理能力,也让世界陷入``合法性危机'':我们无法就什么是``共同的善''达成共识。
  因此,我们需要寻找一种能够跨越文明差异、能够支撑全球合作的\textbf{共同价值}。

  世界需要一种非排他,可对话,跨文明的共同价值框架,中国应当提出答案。

  在这样的历史点位上,``中国担当''不再是一个国家自我叙事的问题,而是一个``世界如何继续合作''的时代命题。
\end{frame}

\section{共同价值}

\begin{frame}
  \frametitle{普遍性困境}
  ``普世价值''``共同价值''指在现代多元价值体系中不同的价值观的选择。价值作为客体满足主体需求的衡量,价值观的选择也反映了人类的愿景。
  
  从现代的实践上看,传统的普遍主义通常以自身历史经验为全球模板,但这种方式在全球多样性日益凸显的时代显得力不从心。
  
  \begin{itemize}
      \item 它忽视了历史文明的差异性;
      \item 缺乏全球代表性;
      \item 容易演变为文化优越感。
  \end{itemize}
  除此之外,普世价值的理论内核亦存在缺陷。``普世价值''在实践中扮演了一个意识形态幻象核心的角色——一个空无的能指。它的功能不在于其确定的实质内容,而在于激发并引导欲望,使人们为了这个空洞的崇高理念,而接受现实中具体的不公与暴力。

\end{frame}

\begin{frame}
  \frametitle{共在性价值}
  共同价值之作为普世价值的否定之否定
  \begin{itemize}
      \item ``普世价值''作为人原初的理想出现,它宣称自身是一组超越历史、文化、阶级的永恒规范。它首先呈现为一种抽象的、形式的普遍性。其力量源于对一切特殊性的否定与超越,它自我规定为一种无条件的、纯粹的``应然''
      \item 然而,普世价值的概念在实践上是缺失的。``普世价值''为实现自身,必须对自身进行规定和特殊化,但不同的解释可能彼此不同,而承认其一必然导致霸权,这就与其所声称的普遍性相违背了。
      \item 将普世价值必然导致的霸权性扬弃,便开辟了一个随实践运动着的空间,而这个空间便包含了合理的可接受且可实践的共同价值。
  \end{itemize}
  因此,中国的``共同价值''不是去构建一种新的普遍主义,而是强调:
  \begin{itemize}
      \item 共性来自人类相互依存的现实,而不是来自单一文明的延伸;
      \item 差异不是障碍,而是构成共识的起点;
      \item 文明之间可以形成``重叠共识'',而不是统一化的价值模板。
  \end{itemize}
\end{frame}

\begin{frame}
  \frametitle{共在性价值}
  因此我们今天讲的``共同价值'',不是要把世界塑造成单一模式,而是寻找不同文明之间可以共享的重叠共识:
  \begin{itemize}
      \item 对和平的重视,
      \item 对发展的期待,
      \item 对公平与稳定的追求。
\end{itemize}

  这些共识不是某个文明的专利,而是全球相互依存所共同塑造的价值。
\end{frame}

\begin{frame}
  \frametitle{中国思想资源}
  中国文明长期强调多元、整体、共生,为共同价值的提炼提供了深厚土壤:
  \begin{itemize}
      \item ``和而不同'':承认差异、追求和谐,反对同质化;
      \item ``天下''观:以秩序与伦理组织世界,而非以征服或扩张为逻辑;
      \item ``中庸与互利'':重视均衡与合作,强调义利统一;
      \item 传统政治哲学中的``仁政''理念:将共同生活与公共福祉置于价值核心。
  \end{itemize}

  这些思想为全球共识提供了跨文化的表达方式,也为形成新的共同价值创造了思想基础。
\end{frame}

\section{中国担当}

\begin{frame}
  \frametitle{文明逻辑}
  改革开放以来,中国探索了一条不同于西方模式的现代化道路——国家主导、渐进改革、面向发展的全球参与。
  这一道路证明:现代化可以是多样的,不必走单一轨道。
  这本身就是对世界文明的一种贡献,拓展了其他国家的选择空间。
\end{frame}

\begin{frame}
  \frametitle{伦理逻辑}
  中国传统政治文化的核心伦理可以用``关系性''来概括:
  \begin{itemize}
      \item 重视组织与个体之间的责任;
      \item 强调合作胜于竞争;
      \item 信奉利益与道义的结合,而非利益与道义的冲突。
  \end{itemize}

  传统儒家的``义利合一'',``推己及人''等理念,在现代国家政策中得到了不同程度的延续,使中国具备从伦理上理解全球责任的基础。
\end{frame}

\begin{frame}
  \frametitle{结构责任}
  随着中国经济体量在改革开放后逐步扩大(尤其在 1980-2000 年全球化高峰期),中国与世界之间形成高度联系:
  \begin{itemize}
      \item 世界的稳定影响中国的发展;
      \item 中国的发展也影响世界市场、供应链与区域平衡。
  \end{itemize}

  这种结构性嵌入意味着——
  中国无法回避国际责任,也天然拥有推动全球公共利益的能力。
\end{frame}

\section{中国实践}

\begin{frame}
  \frametitle{理念贡献}
  中国系统提出和平发展和科学发展的道路,强调国家之间可以在竞争中共存,在发展中共享利益。
  这一理念成为中国对外关系的重要价值基调,也为其他国家探索非对抗性的现代化路径提供了参考。
\end{frame}

\begin{frame}
  \frametitle{制度贡献}
  中国积极推动和参与多边制度建设,包括:
  \begin{itemize}
      \item 加入恢复联合国合法席位,推动多边主义重建;
      \item 加入世界贸易组织,促进全球贸易体系开放稳定;
      \item 推动上海合作组织,深化安全与经济合作;
      \item 加入世界银行、国际货币基金组织,推动发展议题纳入核心议程;
      \item 推动亚太地区合作,如APEC等;
      \item 推动金砖国家合作;
      \item 与发展中国家的组织77国集团深度合作,``77国集团和中国''称为国际上的焦点。
      \item 扩大与非洲、拉美的合作,如中非合作论坛。
  \end{itemize}

  这些制度贡献强调开放、合作与普遍参与,拒绝排他性结构。
  这些开放多边合作机制为发展中国家参与全球秩序提供了制度平台,强化了南南合作的现实基础。
\end{frame}

\begin{frame}
  \frametitle{公共产品}
  我国的公共产品实践更强调实际发展支持,如:
  \begin{itemize}
      \item 大规模援助与减免最不发达国家债务;
      \item 推动对外基础设施合作与产能合作;
      \item 帮助多国建设农业技术示范中心、派出医疗队、提供教育援助;
      \item 在国际金融危机(2008)中,中国保持投资和进口,为世界经济复苏提供需求动力。
  \end{itemize}

  这些行动并不以意识形态输出为目的,而是以发展作为公共产品直接惠及他国。
\end{frame}

\begin{frame}{行动模式}
中国的行动有以下特征:

  \begin{itemize}
      \item \textbf{以发展为中心}:蒙内铁路将肯尼亚主要城市间运输时间由 12h 缩短至 4h。
      \item \textbf{以合作为前提}:促成沙特与伊朗恢复外交关系,不以阵营划线。
      \item \textbf{以经验为基础}:在塞内加尔推广杂交水稻,亩产由约 200kg 提升至 700kg+。
      \item \textbf{以开放为方向}:RCEP 生效后,中国—东盟贸易更紧密,区域供应链更稳定。
  \end{itemize}

  这种模式为全球南方国家提供了另一种发展可能性。
\end{frame}

\section{全球意义}

\begin{frame}{另一种选择}
中国在西方道路之外提供了一种结合和平、发展与合作的另一条道路:

\begin{itemize}
    \item 以和平、发展、合作为核心。
    \item 支持多元文明共存,例如同时与沙特与伊朗保持合作并促成复交。
    \item 在全球风险中寻求共赢,如推动全球南方在气候资金中的话语权。
\end{itemize}

\end{frame}

\begin{frame}{价值+制度}
中国担当来自''价值理念 + 制度实践''的结合:

\begin{itemize}
    \item \textbf{价值层面}:合作、互利、公共性,如抗疫期间向多国提供物资与疫苗。
    \item \textbf{制度层面}:深度参与联合国、WTO、金砖、上合组织等机制。
\end{itemize}

通过两方面的模式提升执行力,例如在世贸框架促进供应链稳定。

\end{frame}

\begin{frame}{文明角色}
中国在全球秩序变革中的文明作用:

\begin{itemize}
    \item 不输出单一发展模式,而是共享实践经验。
    \item 例如农业技术与脱贫经验在非洲、东南亚被广泛借鉴。
    \item 推动公平多元的全球治理,如倡导''共同但有区别的责任''成为气候共识。
\end{itemize}

\end{frame}

\begin{frame}
  \frametitle{致谢}
  \centerline{\Large 谢谢!}
\end{frame}

\end{document}
